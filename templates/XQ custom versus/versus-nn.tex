\documentclass{standalone}

%%%%% INPUT AND LANGUAGE %%%%%
\usepackage[latin9]{inputenc}
\usepackage[T1]{fontenc}
\usepackage{microtype}
\usepackage{xspace}
\usepackage[english]{babel}

%%%%% GENERAL UTILITIES %%%%%
\usepackage{amsmath}
\usepackage{amssymb}
\usepackage{multicol}
\usepackage{enumitem}
\usepackage{mathtools}
\usepackage{comment}
\usepackage{array}
\usepackage[hidelinks]{hyperref}
\usepackage{fp}

%%%%% FONTS %%%%%
\usepackage{cmbright}
\usepackage[nomath]{lmodern}
\usepackage{inconsolata}
\usepackage{bm}
\DeclareMathAlphabet{\mathsfit}{T1}{\sfdefault}{\mddefault}{\sldefault}
\SetMathAlphabet{\mathsfit}{bold}{T1}{\sfdefault}{\bfdefault}{\sldefault}
\newcommand{\mathbif}[1]{\bm{\mathsfit{#1}}}

%%%%% DRAWINGS %%%%%
\usepackage{tikz}
\usetikzlibrary{
    calc
}
\usepackage{pgfplots}
\usepgfplotslibrary{colorbrewer}
\pgfplotsset{
    compat = 1.15,
    x tick label style = {minimum height = 1pc, inner sep=0pt, outer sep=3pt},
    y tick label style = {minimum height = 1pc, inner sep=0pt, outer sep=5pt},
    every tick/.append style = {black, thick},
}

%%%%% CUSTOM DEFINITIONS %%%%%
\def\mainsource{PARAMETERS2-DATA-FILENAME}
\def\statsource{FITNESS-STATS-FILENAME}
\def\plotwidth{PLOT-WIDTH}
\def\nlowerlimit{N-LOWER-LIMIT}
\def\nupperlimit{N-UPPER-LIMIT}
\def\tlowerlimit{T-LOWER-LIMIT}
\def\tupperlimit{T-UPPER-LIMIT}
\def\mlowerlimit{M-LOWER-LIMIT}
\def\mupperlimit{M-UPPER-LIMIT}
\def\slowerlimit{S-LOWER-LIMIT}
\def\supperlimit{S-UPPER-LIMIT}
\def\nreflowerlimit{NREF-LOWER-LIMIT}
\def\nrefupperlimit{NREF-UPPER-LIMIT}
\def\xlabelsep{-0.10}
\def\ylabelsep{-0.10}
\def\paramsep{1cm}
\def\numsep{5pt}
\def\markopacity{0.3}

\begin{document}%
\pgfdeclareplotmark{plainM}{\fill[draw=none, opacity=\markopacity, fill=Dark2-A] circle [radius=1pt];}%
\pgfdeclareplotmark{plainT}{\fill[draw=none, opacity=\markopacity, fill=Dark2-C] circle [radius=1pt];}%
\pgfdeclareplotmark{plainS}{\fill[draw=none, opacity=\markopacity, fill=Dark2-B] circle [radius=1pt];}%
\pgfdeclareplotmark{plainN}{\fill[draw=none, opacity=\markopacity, fill=Dark2-D] circle [radius=1pt];}%
\pgfdeclareplotmark{plainNref}{\fill[draw=none, opacity=\markopacity, fill=Dark2-E] circle [radius=1pt];}%
\begin{tikzpicture}[
    /pgfplots/scale only axis,
    /pgfplots/width=\plotwidth,
    /pgfplots/height=\plotwidth
    ]
    \pgfplotstableread{\mainsource}\data
    \pgfplotstableread{\statsource}\stats
    \begin{axis}[
        name = t0,
        anchor = south,
        axis on top,
        inner sep = \numsep,
        outer sep = 0pt,
        line width = 1pt,
        tickwidth = {5pt},
        xticklabel pos = top,
        xtick align = inside,
        ytick align = inside,
        scaled y ticks = false,
        ymin = \nlowerlimit,
        ymax = 52,
        xmin = \nlowerlimit,
        xmax = 42,
      ]

      \fill[black!15] (\nlowerlimit, \nlowerlimit) -- (\nupperlimit, 0.9091*\nupperlimit) -- (\nupperlimit, 1.1*\nupperlimit) -- cycle;
      \draw[black!30] (\nlowerlimit, \nlowerlimit) -- (\nupperlimit, \nupperlimit);
      \addplot+[only marks, mark=plainN] table [x = sn, y = n] {\data};
      \draw (\nlowerlimit, 52) node[anchor=north west, inner sep=5pt] {\Large$n$};
      \draw (42, \nlowerlimit) node[anchor=south east, inner sep=10pt, xshift=5pt] {\pgfplotstablegetelem{0}{n}\of\stats\large\pgfplotsretval};
    \end{axis}
    \begin{axis}[
        name = m0,
        at = {($(t0.east) + (\paramsep,0)$)},
        anchor = west,
        axis on top,
        inner sep = \numsep,
        outer sep = 0pt,
        line width = 1pt,
        tickwidth = {5pt},
        xticklabel pos = top,
        xtick align = inside,
        ytick align = inside,
        scaled y ticks = false,
        ymin = \mlowerlimit,
        ymax = \mupperlimit,
        xmin = \mlowerlimit,
        xmax = 50,
      ]

      \fill[black!15] (\mlowerlimit, \mlowerlimit) -- (\mupperlimit, 0.9091*\mupperlimit) -- (\mupperlimit, 1.1*\mupperlimit) -- cycle;
      \draw[black!30] (\mlowerlimit, \mlowerlimit) -- (\mupperlimit, \mupperlimit);
      \addplot+[only marks, mark=plainM] table [x = sM_0, y = M_0] {\data};
      \draw (\mlowerlimit, \mupperlimit) node[anchor=north west, inner sep=5pt] {\Large$M_0$};
      \draw (50, \mlowerlimit) node[anchor=south east, inner sep=10pt, xshift=5pt] {\pgfplotstablegetelem{0}{M0}\of\stats\large\pgfplotsretval};
    \end{axis}

    \foreach \cindex in {MIDDLE-COMPONENT-INDEXES} {
            \FPeval{\cmoindex}{clip(\cindex-1)}
    \begin{loglogaxis}[
        name= t\cindex,
        at = (t\cmoindex.south),
        anchor = north,
        axis on top,
        inner sep = \numsep,
        outer sep = 0pt,
        line width = 1pt,
        tickwidth = {5pt},
        xticklabels = {,,},
        xtick align = inside,
        ytick align = inside,
        scaled y ticks = false,
        ymin = \tlowerlimit,
        ymax = \tupperlimit,
        xmin = 0.1,
        xmax = 5
      ]

      \fill[black!15] (\tlowerlimit,  \tlowerlimit*0.667) -- (\tupperlimit, \tupperlimit*0.667) -- (\tupperlimit, \tupperlimit*1.5) -- (\tlowerlimit,  \tlowerlimit*1.5);
      \draw[black!30] (\tlowerlimit, \tlowerlimit) -- (\tupperlimit, \tupperlimit);
      \addplot+[only marks, mark=plainT] table [x = st_\cindex, y = t_\cindex] {\data};
      \draw (0.1, \tupperlimit) node[anchor=north west, inner sep=5pt] {\Large$t_{\cindex}$};
      \draw (5, \tlowerlimit) node[anchor=south east, inner sep=10pt, xshift=5pt] {\pgfplotstablegetelem{0}{t\cindex}\of\stats\large\pgfplotsretval};
    \end{loglogaxis}
    \begin{axis}[
        name = m\cindex,
        at = (m\cmoindex.south),
        anchor = north,
        axis on top,
        inner sep = \numsep,
        outer sep = 0pt,
        line width = 1pt,
        tickwidth = {5pt},
        xticklabels = {,,},
        xtick align = inside,
        ytick align = inside,
        scaled y ticks = false,
        ymin = \mlowerlimit,
        ymax = \mupperlimit,
        xmin = \mlowerlimit,
        xmax = 50,
      ]

      \fill[black!15] (\mlowerlimit, \mlowerlimit) -- (\mupperlimit, 0.9091*\mupperlimit) -- (\mupperlimit, 1.1*\mupperlimit) -- cycle;
      \draw[black!30] (\mlowerlimit, \mlowerlimit) -- (\mupperlimit, \mupperlimit);
      \addplot+[only marks, mark=plainM] table [x = sM_\cindex, y = M_\cindex] {\data};
      \draw (\mlowerlimit, \mupperlimit) node[anchor=north west, inner sep=5pt] {\Large$M_{\cindex}$};
      \draw (50, \mlowerlimit) node[anchor=south east, inner sep=10pt, xshift=5pt] {\pgfplotstablegetelem{0}{M\cindex}\of\stats\large\pgfplotsretval};
    \end{axis} 
    }

    %KEEP-IF-1C%\def\cindex{LAST-COMPONENT-INDEX}
    %KEEP-IF-1C%    \FPeval{\cmoindex}{clip(\cindex-1)}
    \begin{axis}[
        name= t\cindex,
        at = (t\cmoindex.south),
        anchor = north,
        axis on top,
        inner sep = \numsep,
        outer sep = 0pt,
        width = \plotwidth,
        height = \plotwidth,
        xmode = log,
        ymode = log,
        line width = 1pt,
        tickwidth = {5pt},
        xticklabel pos = bottom,
        xtick align = inside,
        ytick align = inside,
        scaled y ticks = false,
        ymin = \tlowerlimit,
        ymax = \tupperlimit,
        xmin = \tlowerlimit,
        xmax = \tupperlimit,
      ]

      \draw[black!20] (\tlowerlimit, \tlowerlimit) -- (\tupperlimit, \tupperlimit);
      \addplot+[only marks, mark=plainT] table [x = st_\cindex, y = t_\cindex] {\data};
      \draw (\tlowerlimit, \tupperlimit) node[anchor=north west, inner sep=5pt] {\Large$t_{\cindex}$};
    \end{axis}
    \begin{axis}[
        name = m\cindex,
        at = (m\cmoindex.south),
        anchor = north,
        axis on top,
        inner sep = \numsep,
        outer sep = 0pt,
        width = \plotwidth,
        height = \plotwidth,
        line width = 1pt,
        tickwidth = {5pt},
        xticklabel pos = bottom,
        xtick align = inside,
        ytick align = inside,
        scaled y ticks = false,
        ymin = \mlowerlimit,
        ymax = \mupperlimit,
        xmin = \mlowerlimit,
        xmax = \mupperlimit,
      ]

      \draw[black!20] (\mlowerlimit, \mlowerlimit) -- (\mupperlimit, \mupperlimit);
      \addplot+[only marks, mark=plainM] table [x = sM_\cindex, y = M_\cindex] {\data};
      \draw (\mlowerlimit, \mupperlimit) node[anchor=north west, inner sep=5pt] {\Large$M_{\cindex}$};
    \end{axis} 

\end{tikzpicture}%
\end{document} 